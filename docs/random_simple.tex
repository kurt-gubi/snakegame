%%- from "macros.tex" import make_board -%%

\section{Random Bot}
\fasttrack{Choose a direction at random.}

The next bot we’ll write is one which instead of moving in just one direction,
chooses a direction at random to move in.
Go on, try writing it yourself! I’ll wait here until you’re ready.

Got it working? Good work!
But you’ve probably noticed that there’s a problem:
it doesn’t take long for our random bot to die.
But why does it die?
The answer is that once it eats an apple, it then has a tail, and since it
doesn’t know any better, it will happily move into the square where its tail is.

\begin{board}
\hfill
%
\begin{subfigure}{.3\linewidth}
< make_board(['   *', ' A* ', ' *  '])>
\caption{Our intrepid snake heads towards an apple. Next move: \textbf{R}}
\label{brd:random-death:1}
\end{subfigure}
\hfill
%
\begin{subfigure}{.3\linewidth}
< make_board(['*  *', ' aA ', ' *  ']) >
\caption{It has eaten the apple, and now has a tail. Next move: \textbf{L}}
\label{brd:random-death:2}
\end{subfigure}
\hfill
%
\begin{subfigure}{.3\linewidth}
< make_board(['*  *', '    ', ' *  ']) >
\caption{It decided to move left, and ran into itself, oh no!}
\label{brd:random-death:3}
\end{subfigure}
%
\hfill

\caption{The last moves of Random Bot before death.}
\label{brd:random-death}
\end{board}

\pagebreak

By the way, how long was your solution?
If you’re still learning Python, you might like to have a peek at my solution to
this bot, it’s only three lines long.
Hopefully you didn’t write too much more than that!

\pythonfile{random_simple.py}

There are two key things that make my solution work.
The first is the \texttt{random.choice} function,
which returns a random item chosen from a sequence you give it.
The second thing is that a string is a sequence:
it is made up of the characters in it.
So if you write \mint{python}|choice('UDLR')|,
that’s the same as if you had written
\mint{python}|choice(['U', 'D', 'L', 'R'])|.

